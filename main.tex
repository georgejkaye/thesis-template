\documentclass[a4paper, oneside, 12pt, openany]{book}
\pdfoutput=1

\usepackage{packages}
\usepackage{macros}

% Label tables just like equations, theorems, definitions, etc.
%
% NB: This can be confusing if LaTeX does not place the table at the point of
% writing (e.g. for lack of space)!
\numberwithin{equation}{section}
\numberwithin{table}{section}
\makeatletter
\let\c@equation\c@table
\makeatother

% Setting up the coloured environments
%
\newbool{shade-envs}
% This can be used to toggle the coloured environments on or off.
\setboolean{shade-envs}{true}

%%
\ifthenelse{\boolean{shade-envs}}{%
  % Colours are as in Andrej Bauer's notes on realizability:
  % https://github.com/andrejbauer/notes-on-realizability
  \colorlet{ShadeOfPurple}{blue!5!white}
  \colorlet{ShadeOfYellow}{yellow!5!white}
  \colorlet{ShadeOfGreen} {green!5!white}
  \colorlet{ShadeOfBrown} {brown!10!white}
  % But we also shade proofs
  \colorlet{ShadeOfGray}  {gray!10!white}
}
% If we don't want to have shaded environments, then we use a closing symbol
% \lozenge to mark the end of remarks, definitions and examples.
{%
  \declaretheoremstyle[
      spaceabove=6pt,
      spacebelow=6pt,
      bodyfont=\normalfont,
      qed=\(\lozenge\)
  ]{definitionwithbox}
  \declaretheoremstyle[
      headfont=\itshape,
      bodyfont=\normalfont,
      qed=\(\lozenge\)
      ]{remarkwithbox}
}

\ifthenelse{\boolean{shade-envs}}{%
  \declaretheorem[sibling=equation]{theorem}
  \declaretheorem[sibling=theorem]{lemma,proposition,corollary}
  \declaretheorem[sibling=theorem,style=definition]{definition}
  \declaretheorem[sibling=theorem,style=definition]{example}
  \declaretheorem[sibling=theorem,style=remark]{remark}
  % Now we set the shading using the tcolorbox package.
  %
  % The related thmtools' option "shaded" and the package mdframed seem to have
  % issues: the former does not allow for page breaks in shaded environments and
  % the latter puts double spacing between two shaded environments.
  \tcbset{shadedenv/.style={
      colback={#1},
      frame hidden,
      enhanced,
      breakable,
      boxsep=0pt,
      left=2mm,
      right=2mm,
      % LaTeX thinks this is too wide (as becomes clear from the many "Overfull
      % \hbox" warnings, but optically it looks spot on.
      add to width=1.1mm,
      enlarge left by=-0.6mm}
  }
  %
  \tcolorboxenvironment{theorem}    {shadedenv={ShadeOfPurple}}
  \tcolorboxenvironment{lemma}      {shadedenv={ShadeOfPurple}}
  \tcolorboxenvironment{proposition}{shadedenv={ShadeOfPurple}}
  \tcolorboxenvironment{corollary}  {shadedenv={ShadeOfPurple}}
  \tcolorboxenvironment{definition} {shadedenv={ShadeOfYellow}}
  \tcolorboxenvironment{example}    {shadedenv={ShadeOfGreen}}
  \tcolorboxenvironment{remark}     {shadedenv={ShadeOfBrown}}
  \tcolorboxenvironment{proof}      {shadedenv={ShadeOfGray}}
}{% Use closing symbols if we don't have colours
  \declaretheorem[sibling=equation]{theorem}
  \declaretheorem[sibling=theorem]{lemma,proposition,corollary}
  \declaretheorem[sibling=theorem,style=definitionwithbox]{definition}
  \declaretheorem[sibling=theorem,style=definitionwithbox]{example}
  \declaretheorem[sibling=theorem,style=remarkwithbox]{remark}
  }
\declaretheorem[sibling=theorem,style=remark,numbered=no]{claim}

% Note that proofs will still have the \qed symbol at the end, even when shaded,
% because we prefer to keep up the tradition.



\begin{document}

\frontmatter

\newgeometry{total={180mm,267mm}} % Make "A thesis submitted...the degree of" fit one line
\begin{titlepage}
\begin{center}
  \vspace*{\stretch{0.5}}

  \large % Default size for the title page

  {\Huge\textsc{The Title of a PhD Thesis}\par}

  \vspace{\stretch{0.2}}

  by

  \vspace{\stretch{0.2}}

  {\huge\textsc{Jane Doe}}

  \vspace{\stretch{0.5}}

  A thesis submitted to the University of Foo for the degree of\\
  \textsc{Doctor of Philosophy}

  \vfill

  \flushright
  School of Bar \\
  College of Baz and Quz \\
  University of Foo \\
  January 2023

\end{center}
\end{titlepage}
\restoregeometry%

\chapter{Abstract}

\lipsum[1]


%%% Local Variables:
%%% mode: latexmk
%%% TeX-master: "../main"
%%% End:

\chapter{Acknowledgements}\label{chap:acknowledgements}
\markboth{Acknowledgements}{Acknowledgements}

\lipsum[1-4]

%%% Local Variables:
%%% mode: latexmk
%%% TeX-master: "../main"
%%% End:

\setcounter{tocdepth}{2}
\tableofcontents

\mainmatter%

\chapter{Some math environments}

\lipsum[5]

A few citations~\cite{SGA1,LaTeX}, a url \url{https://latex.org} and a forward
reference to a result, see~\cref{thm} below.

\section{Definitions, theorems and proofs}

\begin{definition}\index{simplicial!hom-object}\index{simplicial!mor-object}
  A \emph{\(C\)-simplicial hom-object} a lift of a monoid \(B \to C\) and a
  \emph{\(C\)-simplicial mor-object} is...
\end{definition}

\begin{lemma}
  The formula \(\sqrt a \cdot \sqrt b = \sqrt{ab}\) holds.
  %
  \nomenclature[sqrt]{$\sqrt{}$}{square root}
\end{lemma}

\begin{proposition}\index{functor}\index{foo}\index{bar}
  If a functor \(F : \mathcal C \to \mathcal D\) is foo, then it is bar.
  \nomenclature[C]{$\mathcal C$}{a category}
\end{proposition}

\section{Definitions, theorems and proofs}

\lipsum[4]

\subsection{Subsection 1}

\lipsum[5]

\subsection{Subsection 2}

\lipsum[7]

\begin{theorem}\label{thm}
  A map \(f : A \to B\) is foo if and only if all its fibres are bar.
\end{theorem}
\begin{proof}
  \lipsum[7]
  \[
    \int_0^\infty e^{-\alpha x^2} \mathrm{d}x =
    \frac12\sqrt{\int_{-\infty}^\infty e^{-\alpha x^2}}
    \mathrm{d}x\int_{-\infty}^\infty e^{-\alpha y^2}\mathrm{d}y =
    \frac12\sqrt{\frac{\pi}{\alpha}}
  \]
  \lipsum[8]
\end{proof}

\begin{corollary}
  This corollary is some immediate consequence of the theorem.
\end{corollary}

\begin{remark}
  \lipsum[1-2]
\end{remark}

\begin{example}
  \lipsum[5]
\end{example}

See~\cref{table} below.
\begin{table}[h]%
  \centering
  \begin{tabular}{llp{4cm}l}\toprule
    & Name & Affiliation & Email address \\\midrule
    1. & Dr.~Abc Def
       & University of Foo
       & \href{mailto:abc.def@foo.ac.uk}{\texttt{abc.def@foo.ac.uk}} \\
    2. & Prof.~Qwerty Xyz
       & University of Bar
       & \href{mailto:qwerty.xyz@bar.ac.uk}{\texttt{qwerty.xyz@bar.edu}}
  \end{tabular}
  \caption{The caption of a table.}
  \label{table}
\end{table}


%%% Local Variables:
%%% mode: latexmk
%%% TeX-master: "../main"
%%% End:

\chapter{Conclusion}

\lipsum[7-12]


%%% Local Variables:
%%% mode: latexmk
%%% TeX-master: "../main"
%%% End:


\backmatter%
\printbibliography[heading=bibintoc]%
\printnomenclature%
\printindex

\end{document}
